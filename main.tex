%%%%%%%%%%%%%%%%%%%%%%%%%%%%%%%%%%%%%%%%%%%%%%%%%%%%%%%%%%%%%%%%%%%%%%%%%%%%%%%
%                                                                             %
% Anwendungsbeispiel für das Beamer-Template TU Chemnitz                      %
% (c) Mario Haustein (mario.haustein@hrz.tu-chemnitz.de), 2013-2019           %
%                                                                             %
%%%%%%%%%%%%%%%%%%%%%%%%%%%%%%%%%%%%%%%%%%%%%%%%%%%%%%%%%%%%%%%%%%%%%%%%%%%%%%%

\usepackage[T1]{fontenc}
\usepackage[utf8]{inputenc}
\usepackage[ngerman]{babel}


% TUC-Templates laden.
\usetheme{tuc2019}
\mode<article>{\usepackage{beamerarticletuc2019}}


%
% Weitere Anpassungen nach Bedarf.
%

%% Navigationsleiste deaktivieren.
\setbeamertemplate{navigation symbols}{}

%% Mathematische Sätze nummerieren.
%\setbeamertemplate{theorems}[numbered]

%% Titelzeile fett und in Standardfarbe formatieren.
% \setbeamercolor{frametitle}{parent=normal text}
% \setbeamerfont{frametitle}{series=\bfseries}

% Metadaten
\newcommand{\Title}{Benutzerfreundlichkeit fünf verschiedener Zwei-Faktor-Authentifizierungsmethoden}
\newcommand{\ShortTitle}{Zwei-Faktor-Authentifizierungsmethoden}
\title[\ShortTitle]{\Title}
\subtitle{Hauptseminar Medieninformatik}
\author{Joshua Jeschek}
\date{11.06.2022}
\institute[TUC]{TU Chemnitz}
\titlegraphic{\includegraphics[height=0.19\textheight]{tuc2019/logo/tuc_green}}
\tucurl{https://hauptseminar.jeschek.eu}


\begin{document}
\tucthreeheadlines{}
\frame{\titlepage}

\tuctwoheadlines{}
\section*{\ShortTitle}
\subsection{Einleitung}
\section*{\ShortTitle}
\subsection*{Einleitung}
\begin{frame}
  \begin{center}
    \begin{figure}
      \begin{tikzpicture}
        \begin{scope} [fill opacity=.4]
          \node[circle,draw,fill=gray,minimum width=4cm,anchor=-35]
          (know) at (0,1) {
            \includegraphics[width=0.2\textwidth]{password}};
          \node[circle,draw,fill=gray,minimum width=4cm,anchor=215]
          (have) at (0,1) {
            \includegraphics[height=0.2\textwidth]{phone}};
          \node[circle,draw,fill=gray,minimum width=4cm,anchor=90]
          (are) at (0,1) {
            \includegraphics[height=0.2\textwidth]{touch-id}};
        \end{scope}
      \end{tikzpicture}
      \caption{Different possibilities for 2FA Methods
        \endnote{\url{https://www.pngegg.com/en/png-eyxan}}
        \endnote{\url{https://images.vexels.com/media/users/3/157570/isolated/lists/4b39b362c76ea5a00de62f8ff839b5ed-einfaches-smartphone-symbol.png}}
        \endnote{\url{https://cdn4.iconfinder.com/data/icons/apple-touch-id/512/Touch_ID-512.png}}
      }
    \end{figure}
  \end{center}

  \note{
    \begin{itemize}
      \item seem to already know what 2 Factor Authentication is\ldots
      \item 2 Factor Authentication, also abreviated with 2FA requires user to present 2 of 3 authentication factors
      \item These are
            \begin{itemize}
              \item something you know \textrightarrow{} most of the times a password
              \item something you have \textrightarrow{} a phone or an additional device
              \item something you are \textrightarrow{} fingerprint or iris scan
            \end{itemize}
      \item Combination of 2 different factors \textrightarrow{}
    \end{itemize}
  }
\end{frame}


\subsection{Gliederung}
\frame{\frametitle{Gliederung}\tableofcontents[hideallsubsections]}

%
% Inhalt
%

\section{Fünf Zwei-Faktor-Authentifizierungsmethoden}
\section{Five Two-Factor Authentication Methods}

\subsection{SMS}
\begin{frame}
  \frametitle{\currentsectionname}
  \begin{itemize}
    \item Confirmation code is sent as an SMS message
    \item Input of code into app
    \item Most widely used
    \item Some usability and security issues, but simple setup
  \end{itemize}
  \note{
    Usability Problems:
    \begin{itemize}
      \item delayed or no delivery
      \item user error when entering code
    \end{itemize}
    Security Problems:
    \begin{itemize}
      \item no encryption \textrightarrow{} Man In The Middle Attack
      \item SIM-Swapping \textrightarrow{} way to steal phone number, so codes are sent to wrong user
      \item Brute-Force attacks are possible, since codes are quite short
    \end{itemize}
  }
\end{frame}

\subsection{TOTP}
\begin{frame}
  \frametitle{\currentsectionname}
  \begin{itemize}
    \item \textit{Time-based One-time Password}
    \item User scans QR code, scanner generates confirmation code
    \item Confirmation code is entered and verified by server
    \item Code has limited lifespan
    \item Special scanner or smartphone required
  \end{itemize}

  \note{
    \begin{itemize}
      \item requires no internet access or phone reception\textrightarrow{} less attack surface
    \end{itemize}
    Usability Problems:
    \begin{itemize}
      \item relatively complicated to setup
      \item user error when entering code
    \end{itemize}
    Security:
    \begin{itemize}
      \item Encrypted \textrightarrow{} better than SMS
    \end{itemize}
  }
\end{frame}

\subsection{Pre-Generated Codes}
\begin{frame}
  \frametitle{\currentsectionname}
  \begin{itemize}
    \item List of codes is pre-generated
    \item Approximately eight characters long
    \item Codes must be stored securely
  \end{itemize}

  \note{
    \begin{itemize}
      \item relatively long codes may lead to errors when storing or entering them
      \item Users need to store codes securely, can be stolen
      \item codes valid for a long time \textrightarrow{} brute force possible
    \end{itemize}
  }
\end{frame}

\subsection{Push Notifications}
\begin{frame}
  \frametitle{\currentsectionname}
  \begin{enumerate}
    \item User receives notification on smartphone
    \item Option to confirm or deny login attempt
  \end{enumerate}
  \begin{itemize}
    \item Requires smartphone and typically specific app
    \item No input of code required
    \item Fast and convenient
  \end{itemize}

  \note{
    \begin{itemize}
      \item notification needs to be sent to right device \rightarrow{} ideally the device nearest to the user
      \item no need to enter code \textrightarrow{} less prone to error
    \end{itemize}
  }
\end{frame}

\subsection{U2F Security Tokens}
\begin{frame}
  \frametitle{\currentsectionname}
  \begin{columns}
    \begin{column}{0.7\textwidth}
      \begin{itemize}
        \item \textit{Universal 2nd Factor}
        \item User uses a U2F-compatible hardware device, such as a security key, to authenticate
        \item Fast and secure
        \item Requires U2F hardware device
      \end{itemize}
    \end{column}
    \begin{column}{0.3\textwidth}
      \begin{figure}
        \includegraphics[height=\linewidth]{ubikey}
        \caption{YubiKey\endnote{\url{https://media.yubico.com/media/catalog/product/5/n/5nfc_hero_2021.png}}}
      \end{figure}
    \end{column}
  \end{columns}

  \note{
    \begin{itemize}
      \item conceived as a very secure, but still usable method
            \begin{itemize}
              \item[\textrightarrow] that's why very interesting later, when we look at the study results
            \end{itemize}
      \item Security Risk: Loss of U2F token
            \begin{itemize}
              \item[\textrightarrow] important to have a backup safely stored away
            \end{itemize}
    \end{itemize}
  }
\end{frame}


\tucthreeheadlines{}
\section{}
\frame<beamer| trans>{\begin{center}\scalebox{2}{Vielen Dank für eure Aufmerksamkeit.}\end{center}}

% \begin{onlyenv}<beamer| trans>
% \appendix
% \newcounter{finalframe}
% \setcounter{finalframe}{\value{framenumber}}

% %
% % "Spare-Slides" hier einsetzen.
% %

% \setcounter{framenumber}{\value{finalframe}}
% \end{onlyenv}

\tuctwoheadlines{}
\section*{\ShortTitle}
\subsection{Literatur}
% \begin{frame}
\frametitle{Literatur}

\begin{thebibliography}{xxx}
\bibitem{reese2019}
    Reese, K., Smith, T., Dutson, J., Armknecht, J., Cameron, J. \& Seamons, K., 2019 \---
    A Usability Study of Five Two-Factor Authentication Methods. SOUPS @ USENIX Security Symposium.
\newblock{\url{https://www.usenix.org/conference/soups2019/presentation/reese}}
% \newblock{}
\end{thebibliography}
\end{frame}

\begin{frame}
\frametitle{Literatur}
\bibliographystyle{plain}
\bibliography{refs}
\end{frame}

\end{document}
