\usepackage[T1]{fontenc}
\usepackage[utf8]{inputenc}
\usepackage[ngerman]{babel}
\usepackage{tikz} % venn diagram
\usepackage{endnotes} % image sources
\usepackage[nobiblatex]{xurl} % url line breaks
\usepackage{pgfplotstable} % csv to table
\usepackage{ tipa } % right up arrow

% currentname
\usepackage{nameref}
\makeatletter
\newcommand*{\currentsectionname}{\@currentlabelname}
\makeatother

% load TUC templates, set color scheme to IF
\usetheme[fakcolor=if]{tuc2019}
\mode<article>{\usepackage{beamerarticletuc2019}}


% customization
\graphicspath{ {./images/} }
\bibliographystyle{alpha}
\nocite{*}

% deactivate navigation
\setbeamertemplate{navigation symbols}{}

% metadata
\newcommand{\Title}{Benutzerfreundlichkeit fünf verschiedener Zwei-Faktor-Authentifizierungsmethoden}
\newcommand{\ShortTitle}{Zwei-Faktor-Authentifizierungsmethoden}
\title[\ShortTitle]{\Title}
\subtitle{Hauptseminar Medieninformatik}
\author{Joshua Jeschek}
\date{11.06.2022}
\institute[TUC]{TU Chemnitz}
\titlegraphic{\includegraphics[height=0.19\textheight]{tuc2019/logo/tuc_green}}
\tucurl{https://hauptseminar.jeschek.eu}


\begin{document}
\tucthreeheadlines{}
\frame{\titlepage}

\tuctwoheadlines{}
\section*{\ShortTitle}
\subsection*{Einleitung}
\begin{frame}
  \begin{center}
    \begin{figure}
      \begin{tikzpicture}
        \begin{scope} [fill opacity=.4]
          \node[circle,draw,fill=gray,minimum width=4cm,anchor=-35]
          (know) at (0,1) {
            \includegraphics[width=0.2\textwidth]{password}};
          \node[circle,draw,fill=gray,minimum width=4cm,anchor=215]
          (have) at (0,1) {
            \includegraphics[height=0.2\textwidth]{phone}};
          \node[circle,draw,fill=gray,minimum width=4cm,anchor=90]
          (are) at (0,1) {
            \includegraphics[height=0.2\textwidth]{touch-id}};
        \end{scope}
      \end{tikzpicture}
      \caption{Different possibilities for 2FA Methods
        \endnote{\url{https://www.pngegg.com/en/png-eyxan}}
        \endnote{\url{https://images.vexels.com/media/users/3/157570/isolated/lists/4b39b362c76ea5a00de62f8ff839b5ed-einfaches-smartphone-symbol.png}}
        \endnote{\url{https://cdn4.iconfinder.com/data/icons/apple-touch-id/512/Touch_ID-512.png}}
      }
    \end{figure}
  \end{center}

  \note{
    \begin{itemize}
      \item seem to already know what 2 Factor Authentication is\ldots
      \item 2 Factor Authentication, also abreviated with 2FA requires user to present 2 of 3 authentication factors
      \item These are
            \begin{itemize}
              \item something you know \textrightarrow{} most of the times a password
              \item something you have \textrightarrow{} a phone or an additional device
              \item something you are \textrightarrow{} fingerprint or iris scan
            \end{itemize}
      \item Combination of 2 different factors \textrightarrow{}
    \end{itemize}
  }
\end{frame}


\subsection*{Gliederung}
\frame{\frametitle{\currentsectionname}\tableofcontents}

%
% CONTENT
%

\section{Five Two-Factor Authentication Methods}

\subsection{SMS}
\begin{frame}
  \frametitle{\currentsectionname}
  \begin{itemize}
    \item Confirmation code is sent as an SMS message
    \item Input of code into app
    \item Most widely used
    \item Some usability and security issues, but simple setup
  \end{itemize}
  \note{
    Usability Problems:
    \begin{itemize}
      \item delayed or no delivery
      \item user error when entering code
    \end{itemize}
    Security Problems:
    \begin{itemize}
      \item no encryption \textrightarrow{} Man In The Middle Attack
      \item SIM-Swapping \textrightarrow{} way to steal phone number, so codes are sent to wrong user
      \item Brute-Force attacks are possible, since codes are quite short
    \end{itemize}
  }
\end{frame}

\subsection{TOTP}
\begin{frame}
  \frametitle{\currentsectionname}
  \begin{itemize}
    \item \textit{Time-based One-time Password}
    \item User scans QR code, scanner generates confirmation code
    \item Confirmation code is entered and verified by server
    \item Code has limited lifespan
    \item Special scanner or smartphone required
  \end{itemize}

  \note{
    \begin{itemize}
      \item requires no internet access or phone reception\textrightarrow{} less attack surface
    \end{itemize}
    Usability Problems:
    \begin{itemize}
      \item relatively complicated to setup
      \item user error when entering code
    \end{itemize}
    Security:
    \begin{itemize}
      \item Encrypted \textrightarrow{} better than SMS
    \end{itemize}
  }
\end{frame}

\subsection{Pre-Generated Codes}
\begin{frame}
  \frametitle{\currentsectionname}
  \begin{itemize}
    \item List of codes is pre-generated
    \item Approximately eight characters long
    \item Codes must be stored securely
  \end{itemize}

  \note{
    \begin{itemize}
      \item relatively long codes may lead to errors when storing or entering them
      \item Users need to store codes securely, can be stolen
      \item codes valid for a long time \textrightarrow{} brute force possible
    \end{itemize}
  }
\end{frame}

\subsection{Push Notifications}
\begin{frame}
  \frametitle{\currentsectionname}
  \begin{enumerate}
    \item User receives notification on smartphone
    \item Option to confirm or deny login attempt
  \end{enumerate}
  \begin{itemize}
    \item Requires smartphone and typically specific app
    \item No input of code required
    \item Fast and convenient
  \end{itemize}

  \note{
    \begin{itemize}
      \item notification needs to be sent to right device \rightarrow{} ideally the device nearest to the user
      \item no need to enter code \textrightarrow{} less prone to error
    \end{itemize}
  }
\end{frame}

\subsection{U2F Security Tokens}
\begin{frame}
  \frametitle{\currentsectionname}
  \begin{columns}
    \begin{column}{0.7\textwidth}
      \begin{itemize}
        \item \textit{Universal 2nd Factor}
        \item User uses a U2F-compatible hardware device, such as a security key, to authenticate
        \item Fast and secure
        \item Requires U2F hardware device
      \end{itemize}
    \end{column}
    \begin{column}{0.3\textwidth}
      \begin{figure}
        \includegraphics[height=\linewidth]{ubikey}
        \caption{YubiKey\endnote{\url{https://media.yubico.com/media/catalog/product/5/n/5nfc_hero_2021.png}}}
      \end{figure}
    \end{column}
  \end{columns}

  \note{
    \begin{itemize}
      \item conceived as a very secure, but still usable method
            \begin{itemize}
              \item[\textrightarrow] that's why very interesting later, when we look at the study results
            \end{itemize}
      \item Security Risk: Loss of U2F token
            \begin{itemize}
              \item[\textrightarrow] important to have a backup safely stored away
            \end{itemize}
    \end{itemize}
  }
\end{frame}

\section{Usability Study (BYU 2019)}

\subsection*{Overview}
\begin{frame}
  \frametitle{Usability Study (BYU 2019)}

  \begin{itemize}
    \item Conducted at Brigham Young University, Utah
    \item Two independent studies:
          \begin{enumerate}
            \item Two-week study on the use of 2FA methods
                  \begin{itemize}
                    \item Login to online banking app
                    \item 12 times over course of two weeks
                  \end{itemize}
            \item Study on the setup of different 2FA methods
          \end{enumerate}
    \item Evaluation of time and usability
  \end{itemize}
  \note{
    \begin{itemize}
      \item Two parts, day-to-day usage and only setup process
      \item day-to-day
            \begin{itemize}
              \item no remember me feature
              \item every participant only one method
            \end{itemize}
      \item setup process
            \begin{itemize}
              \item every participant sets up every method
              \item can compare it
              \item tested seperately \textrightarrow maybe only setup needs to be improved or vice versa
            \end{itemize}
    \end{itemize}
  }

\end{frame}

\subsection{Day-to-Day Use}

\begin{frame}
  \frametitle{Time to Authenticate}

  \begin{itemize}
    \item Time from successful password entry to authentification
  \end{itemize}
  \begin{figure}[c]
    \includegraphics[height=0.8\textheight]{authentication-time}\cite{reese2019}
  \end{figure}

  \note{
    \begin{itemize}
      \item Erste betrachtete Metrik: Authentifizierungszeit
      \item Zeit, für Passwort benötigt nicht mit eingerechnet
      \item U2F am schnellsten, wahrscheinlich weil Nutzer YubiKey am Gerät stecken haben
      \item am zweitschnellsten Push-Benachrichtigungen
      \item Andere 3 Methoden: Zeit zum Eintippen der Zahlen benötigt, verlangsamt Prozess
      \item SMS: Zeit zum Versenden kommt dazu % chktex 13
      \item Codes müssen rausgesucht werden, nicht unbedingt immer zur Hand
      \item \begin{itemize}
              \item[\textrightarrow] wahrscheinlich große Varianz aufgrund von verschiedenen Aufbewahrungsmöglichkeiten
            \end{itemize}
    \end{itemize}
  }

\end{frame}

\begin{frame}
  \frametitle{Usability of Authentication}

  \begin{itemize}
    \item Assessment of perceived usability according to standard scale SUS\,\includegraphics[height=0.25\baselineskip]{sus}
          {\tiny\textcolor{white}{\endnote{\url{https://borderpolar.com/wp-content/uploads/2021/06/red-among-us-png-842x1024.png.webp}}}} % chktex 29
  \end{itemize}
  \begin{figure}[c]
    \includegraphics[height=0.76\textheight]{authentication-sus}\cite{reese2019}
  \end{figure}

  \note{
    \begin{itemize}
      \item Passwort als Kontrollwert
      \item andere Werte alle schlechter, fügen ja nur weitere Schritte hinzu
      \item obwohl TOTP am drittlangsamsten, höchster Score unter 2FA-Methoden
            \begin{itemize}
              \item Google App wurde benutzt, Implementierung wahrscheinlich nutzerfreundlich
              \item Manche Personen haben sich beschwert, dass die Codes zu schnell ablaufen zum Eingeben
              \item Erklärt auch längere Zeit aus vorheriger Folie
            \end{itemize}
      \item Auch interessant: U2F hatte sehr große Varianz, schnitt am schlechtesten ab, obwohl zeitlich am schnellsten
            \begin{itemize}
              \item Zeit nicht alles, muss auch komfortabel sein
              \item große Unterschiede durch verschiedene Möglichkeiten YubiKey aufzubewahren
              \item[\textrightarrow] nicht alle wollen extra Gerät, ist kein Handy was man so oder so hat
            \end{itemize}
    \end{itemize}
  }

\end{frame}

\subsection{Setup of 2FA}
\begin{frame}
  \frametitle{Time to Setup}

  \begin{itemize}
    \item Time between start of setup and completion
  \end{itemize}
  \begin{figure}[c]
    \includegraphics[height=0.8\textheight]{setup-time}\cite{reese2019}
  \end{figure}

  \note{
    \begin{itemize}
      \item Einrichtungszeit für Codes am geringsten,
            da Zeit zum Abspeichern / Ausdrucken / wie auch immer aufbewahren nicht mit betrachtet wurde
            \begin{itemize}
              \item[\textrightarrow] Zeiten wären zu unterschiedlich voneinander gewesen, da schon von Drucker abhängt
            \end{itemize}
      \item Push und SMS am zweitschnellsten, Angabe von Telefonnummer und Verbindung mit Push-App schnell
      \item TOTP und U2F hatten zwei fehlgeschlagene Einrichtungen und waren am langsamsten
      \item Langsame Einrichtung von U2F trotzdem noch bei ca.\ einer Minute
            \begin{itemize}
              \item[\textrightarrow] durch schnelle tagtägliche Benutzung vielleicht zu Entschuldigen
              \item Allerdings muss Nutzer bereits U2F Gerät wie YubiKey besitzen
            \end{itemize}
    \end{itemize}
  }

\end{frame}

\begin{frame}
  \frametitle{Setup Usability}

  \begin{itemize}
    \item Single Ease Question (SEQ), scale from 1 (very difficult) to 7 (very easy)
  \end{itemize}
  \begin{figure}[c]
    \includegraphics[height=0.7\textheight]{setup-seq}~\cite{reese2019}
  \end{figure}

  \note{
    \begin{itemize}
      \item Nutzer bewerten Usability über Single Ease Question
            \begin{itemize}
              \item Eine Frage: ``Wie Einfach war es, die Einrichtung abzuschließen?''
              \item Bewertung auf Skala von 1 bis 7, von sehr schwer bis sehr einfach
              \item Wahl von SEQ, um Ermüdung vorzubeugen, da alle Teilnehmer*innen alle Methoden eingerichtet haben
            \end{itemize}
      \item mit zeitlicher Erfassung übereinstimmend Push und SMS am einfachsten eingestuft
      \item Vermutung:
            \begin{itemize}
              \item Bekannheit von SMS verbessert Ergebnis von SMS
              \item die meisten kennen U2F und TOTP nicht, weshalb die Scores schlechter sein könnten
            \end{itemize}
    \end{itemize}
  }

\end{frame}


\tucthreeheadlines{}
\frame<beamer| trans>{\begin{center}\scalebox{2}{Vielen Dank für eure Aufmerksamkeit.}\end{center}}

% \begin{onlyenv}<beamer| trans>
% \appendix
% \newcounter{finalframe}
% \setcounter{finalframe}{\value{framenumber}}

% %
% % "Spare-Slides" hier einsetzen.
% %

% \setcounter{framenumber}{\value{finalframe}}
% \end{onlyenv}

\tuctwoheadlines{}
\section*{\ShortTitle}
\subsection*{Literatur}
\begin{frame}
    \frametitle{\currentsectionname}
    \bibliography{refs}
\end{frame}

\subsection*{Bildquellen}
\begin{frame}
    \frametitle{\currentsectionname}
    \theendnotes{}
\end{frame}

\end{document}
