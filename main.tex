%%%%%%%%%%%%%%%%%%%%%%%%%%%%%%%%%%%%%%%%%%%%%%%%%%%%%%%%%%%%%%%%%%%%%%%%%%%%%%%
%                                                                             %
% Anwendungsbeispiel für das Beamer-Template TU Chemnitz                      %
% (c) Mario Haustein (mario.haustein@hrz.tu-chemnitz.de), 2013-2019           %
%                                                                             %
%%%%%%%%%%%%%%%%%%%%%%%%%%%%%%%%%%%%%%%%%%%%%%%%%%%%%%%%%%%%%%%%%%%%%%%%%%%%%%%

\usepackage[T1]{fontenc}
\usepackage[utf8]{inputenc}
\usepackage[ngerman]{babel}



% TUC-Templates laden.
\usetheme{tuc2019}
\mode<article>{\usepackage{beamerarticletuc2019}}


%
% Weitere Anpassungen nach Bedarf.
%

%% Navigationsleiste deaktivieren.
%\setbeamertemplate{navigation symbols}{}

%% Mathematische Sätze nummerieren.
%\setbeamertemplate{theorems}[numbered]

%% Titelzeile fett und in Standardfarbe formatieren.
%\setbeamercolor{frametitle}{parent=normal text}
%\setbeamerfont{frametitle}{series=\bfseries}



% Metadaten
\title[Kurztitel]{Titel}
\subtitle{Lehrveranstaltung bzw. Untertitel}
\author{Autor}
\date{Datum bzw. Semester}
\institute[TUC]{TU Chemnitz}
\titlegraphic{\includegraphics[height=0.2\textheight]{tuc2019/logo/tuc_green}}
\tucurl{http://www.tu-chemnitz.de/urz/}



\begin{document}
\tucthreeheadlines
\frame{\titlepage}
\note{Anmerkungen}

\tuctwoheadlines
\frame{\frametitle{Gliederung}\tableofcontents}
\note{\begin{itemize}
\item Weitere Anmerkungen \dots
\item \dots\ in Stichpunkten \dots
\item \dots\ formuliert.
\end{itemize}}

%
% Inhalt
%

\section{}
\frame<beamer| trans>{\Huge\begin{center}\scalebox{2}{\texttt{\textbackslash endinput}}\end{center}}

\begin{onlyenv}<beamer| trans>
\appendix
\newcounter{finalframe}
\setcounter{finalframe}{\value{framenumber}}

%
% "Spare-Slides" hier einsetzen.
%

\setcounter{framenumber}{\value{finalframe}}
\end{onlyenv}
\end{document}
