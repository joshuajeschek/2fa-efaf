\frame{\tableofcontents[sectionstyle=show/hide,subsectionstyle=show/show/hide]}

\subsection{SMS}
\begin{frame}
    \frametitle{SMS}
    \begin{enumerate}
        \item Bestätigungscode wird als SMS-Nachricht gesendet
        \item Eingabe des Codes in App
    \end{enumerate}
    \begin{itemize}
        \item am weitesten verbreitet\cite{reese2019}
        \item einige Usability- und Sicherheitsprobleme, aber einfaches Setup
    \end{itemize}
    \note{
        Usability-Probleme:\cite{reese2019}
        \begin{itemize}
            \item verzögerte Zustellung
            \item kein Empfang
            \item Fehler bei Eingabe des Codes
        \end{itemize}
        Sicherheits-Probleme:\cite{reese2019}
        \begin{itemize}
            \item keine Verschlüsselung -> Man In The Middle
            \item SIM-Swapping
            \item Server muss Code sicher speichern
            \item Brute-Force Angriffe sind nicht ausgeschlossen
        \end{itemize}
    }
\end{frame}

\subsection{TOTP}
\begin{frame}
    \frametitle{TOTP}
    \begin{enumerate}
        \item Nutzer scannt QR-Code, Scanner generiert Bestätigungscode
        \item Bestätigungscode wird eingegeben und vom Server überprüft
    \end{enumerate}
    \begin{itemize}
        \item Code hat beschränkte Lebensdauer
        \item Spezieller Scanner oder Smartphone benötigt
    \end{itemize}
    \note{
        \begin{itemize}
            \item benötigt kein Internet oder Netz
        \end{itemize}
        Usability:\cite{reese2019}
        \begin{itemize}
            \item relative kompliziertes Setup
            \item nicht alle haben Smartphone
            \item Fehler bei Eingabe des Codes
        \end{itemize}
        Sicherheit:\cite{reese2019}
        \begin{itemize}
            \item Verschlüsselung \textrightarrow{} besser als SMS
            \item Server und Client müssen Code sicher speichern
        \end{itemize}
    }
\end{frame}

\subsection{Vor-generierte Codes}

\subsection{Push Benachrichtigungen}

\subsection{U2F Sicherheitsschlüssel}
