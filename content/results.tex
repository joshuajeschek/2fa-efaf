\section{Schlussfolgerungen und andere Studien}

\subsection*{Studienergebnisse}
\begin{frame}
    \frametitle{\currentsectionname}

    \begin{itemize}
        \item keine klaren Gewinner \textrightarrow{} jede Methode kann verbessert werden
        \item U2F ist zwar sicher, aber noch nicht gebrauchstauglich genug
        \item verschiedene Anforderungen an Sicherheit und Komfort
        \item abhängig von Anwendung und Nutzer
    \end{itemize}

    \note{
        \begin{itemize}
            \item Vorgenerierte Codes relativ unsicher, langsam aber einfach in der Benutzung,
                schwierig Einrichtung zu vergleichen (Drucker, Aufbewahrung etc)
            \item Push-Benachrichtigungen sind schnell, und oft gut zu benutzen und einzurichten,
                jedoch große Varianz in Usability der Authentifizierung
            \item SMS weiterhin am weitesten verbreitet, trotz dass sie am unsichersten sind,
                langsam und schwierig in Benutzung, jedoch nicht in der Einrichtung
            \item TOTP schwierig und langwierig in Einrichtung,
                während tagtägliche Gebrauchstauglichkeit trotz relativ großem Zeitaufwand als gut eingeschätzt wurde
            \item U2F mit Ziel entwickelt, sicher und komfortabel zu sein,
                erzielt jedoch nicht die erwarteten Ziele der Gebrauchstauglichkeit (trotz Geschwindigkeit!)
        \end{itemize}
    }

\end{frame}

\subsection*{weitere Studien}
\begin{frame}
    \frametitle{\currentsectionname}

    \begin{itemize}
        \item 2013: A Comparative Usability Study of Two-Factor Authentication\cite{de2013}
        \item 2021: Driving 2FA adoption at scale: Optimizing Two-Factor authentication notification design patterns.\cite{golla2021}
    \end{itemize}
    \newtheorem{fragen}{Fragestellungen}
    \begin{fragen}
        \begin{itemize}
            \item Wieso adoptieren Nutzer 2FA (nicht)?
            \item In welchen Bereichen wollen Nutzer mehr Sicherheit, wo mehr Komfort?
            \item Wie kann die Benutzung von 2FA weiter verbreitet werden?
            \item Wie können unsichere Methoden wie SMS, durch sichere Methoden ersetzt werden?
        \end{itemize}
    \end{fragen}

    \note{
        \begin{itemize}
            \item Studie von 2013 relativ alt, denoch interessant zu Lesen, um zu verstehen,
                    wie Nutzer denken und wie man eine Studie aufbauen könnte
            \item Artikel von 2021 aktueller,
                behandelt wie Nutzer dazu gebracht werden können, 2FA zu benutzen
            \item Fragestellungen\ldots{}
            \item Meinung: sowohl Aufklärung über Sicherheit als auch Usability
                von sicheren Faktoren muss verbessert werden
            \begin{itemize}
                \item Dazu wichtig zu wissen,
                    wie Nutzer denken und was aktuell noch Probleme sind
            \end{itemize}
        \end{itemize}
    }

\end{frame}
