\begin{frame}
    \begin{center}
        % Einleitung \ldots
        %% This block is what you'll need to put in your code where you want your picture.
        \begin{tikzpicture}
            %% You can adjust the opacity here. For venn diagrams it is convenient to have a low opacity so that you can see intersections
            \begin{scope} [fill opacity = .4]
                %% The draw command knows a lot of shapes. To make a rectangle you just need to specify two diagonal corners. Make sure you always have a semicolon at the end of your draw commands, otherwise latex flips out.
                % \draw (-5,5) rectangle (5,-6);
                %% Similarly, you can make a circle by specifying the center and then the radius. You can also add a fill color, but if you're printing in black and white you'll probably want to remove that line.
                \node[circle,
                    draw,
                    fill = gray,
                    rotate = 60,
                    minimum width = 4cm,
                    anchor = 270] (e) at (0,1) {\includegraphics[width=0.2\textwidth,angle=-60]{images/password}};
                \node[circle,
                    draw,
                    fill = gray,
                    rotate = -60,
                    minimum width = 4cm,
                    anchor = 270] (e) at (0,1) {\includegraphics[height=0.2\textwidth,angle=60]{images/phone}};
                \node[circle,
                    draw,
                    fill = gray,
                    rotate = 0,
                    minimum width = 4cm,
                    anchor = 90] (e) at (0,1) {\includegraphics[height=0.2\textwidth]{images/touch-id}};
            \end{scope}
            %% And now you have a venn diagram. Yay!
            % \draw[help lines](-5,5) grid (5,-6);    This line can draw the grid lines to help guide you. I use these when I'm writing the code and then delete this line when I publish the pdf.
        \end{tikzpicture}
    \end{center}
\end{frame}
