% Glossary
\documentclass[10pt,a4paper]{article}

\usepackage{a4wide}
\usepackage{beamerarticle}

\usepackage[T1]{fontenc}
\usepackage[utf8]{inputenc}
\usepackage[english]{babel}
\usepackage{tikz} % venn diagram
\usepackage{enotez} % image sources (instead of endnotes for linking)
\usepackage[nobiblatex]{xurl} % url line breaks
\usepackage{pgfplotstable} % csv to table

% load TUC templates, set color scheme to IF
\usetheme[fakcolor=if]{tuc2019}
\mode<article>{\usepackage{beamerarticletuc2019}}


% customization
\graphicspath{ {./images/} }
\bibliographystyle{alpha}
\nocite{*}

% deactivate navigation
\setbeamertemplate{navigation symbols}{}
\setlength{\tabcolsep}{0.01\linewidth}

% metadata
\newcommand{\Title}{Usability of Five Two-Factor Authentication Methods}
\newcommand{\ShortTitle}{Two-Factor Authentication Methods}
\newcommand{\GlossaryEntry}[4]{#1 & #2 & #3 & #4 \\\hline}
\title[\ShortTitle]{\Title}
\subtitle{EfaF 271-271A - Kurs 3}
\author{Joshua Jeschek}
\date{December 8, 2022}
\institute[TUC]{TU Chemnitz}
\titlegraphic{\includegraphics[height=0.19\textheight]{tuc2019/logo/tuc_green}}
\tucurl{https://www.tu-chemnitz.de}


\begin{document}
\section*{\Title}
\subsection*{Glossary}
\begin{tabular}{ |p{0.2\linewidth}|p{0.1\linewidth}|p{0.4\linewidth}|p{0.3\linewidth}| }
  \hline
  \textbf{New Entry} & \textbf{Word Class} & \textbf{Paraphrase} & \textbf{Sample Sentence} \\
  \hline
  \GlossaryEntry
  {\textbf{2FA} (Two-Factor Authentication)}
  {noun}
  {a security process in which the user provides two different authentication factors to verify their identity}
  {I use 2FA to keep my online accounts safe.}
  \GlossaryEntry
  {\textbf{Pre-generated codes}}
  {noun}
  {list of one-time codes that are generated in advance and do not require an external device}
  {Keep your list of pre-generated codes safe.}
  \GlossaryEntry
  {\textbf{Push Notifications}}
  {noun}
  {a feature of mobile devices that allows an app to send a message to the user}
  {I sometimes miss important emails alongside all these push notifications.}
  \GlossaryEntry
  {\textbf{QR code} (Quick Response code)}
  {noun}
  {two-dimensional barcode used to store data}
  {QR codes can store information such as URLs}
  \GlossaryEntry
  {\textbf{SEQ} (Single Ease Question)}
  {noun}
  {single question which asks users to rate the overall usability on a scale from 1 to 7}
  {The Single Ease Question is a heavily simplified version of the System Usability Scale}
  \GlossaryEntry
  {\textbf{SUS} (System Usability Scale)}
  {noun}
  {standardized questionnaire that is used to measure the usability of a system or product.}
  {The System Usability Scale consists of 10 questions, each of which can be answered on a scale from 1 to 5.}
  \GlossaryEntry
  {\textbf{TOTP} (Time-based One-Time Password)}
  {noun}
  {type of two-factor authentication that uses a combination of a password and a generated code that is valid for a limited time}
  {My time-based one-time password is almost expired, I have to hurry or get a new one.}
  \GlossaryEntry
  {\textbf{U2F} (Universal 2nd Factor)}
  {noun}
  {a type of two-factor authentication that uses a hardware security token}
  {Many online services allow setting up U2F as a Two-Factor Authentication method.}
  \GlossaryEntry
  {\textbf{U2F security token}}
  {noun}
  {a small physical device that the user inserts into a computer's USB port when using U2F}
  {I keep my U2F token on my keychain. This is where my analog and digital keys meet!}
  \GlossaryEntry
  {\textbf{Usability}}
  {noun}
  {ease of use and overall effectiveness of a product or system from the perspective of the user}
  {Usability is a very important factor that should be considered when designing software.}
\end{tabular}
\newpage
\bibliography{refs}
\end{document}
